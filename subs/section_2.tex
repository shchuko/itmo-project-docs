%! Author = vladislav.yaroshchuk
%! Date = 10/10/2021

\pagebreak


\section{Выделение ключевых понятий этапа жизненного цикла ПО из стандартов.}

Каждый из стандартов описывает свою область применения в рамках терминов и понятий.
После анализа выбранных стандартов были выделены следующие основные и дополнительные понятия.


Основные понятия:
\begin{enumerate}
    \item Сопровождение (maintenance)
    \begin{enumerate}
        \item модификации программного продукта в части его кода и документации
        для решения возникающих проблем при эксплуатации или реализации потребностей в
        улучшениях тех или иных характеристик.
        (ГОСТ Р ИСО/МЭК 12207-2010)
        \item modification of existing software while preserving its integrity.
        (SWEBOK v3)
    \end{enumerate}

    \item Сопровождаемость (Maintainability)
    \begin{enumerate}
        \item Набор атрибутов, относящихся к объему работ, требуемых для проведения конкретных изменений (модификаций).
        (ГОСТ Р ИСО/МЭК 9126--93)
        \item the capability of the software product to be modified.
        Modifications may include corrections, improvements or adaptation of the software to changes in environment, and in requirements and functional specifications
        (ISO/IEC 14764:2006)
    \end{enumerate}

    \item Процесс сопровождения (maintenance process):
    \begin{enumerate}
        \item Работы (виды деятельности) и задачи (задания), выполняемые организацией, осуществляющей сопровождение (персоналом сопровождения, сопроводителем).
        (ГОСТ Р ИСО/МЭК 14764--2002)
        \item the totality of activities required to provide cost-effective support to a software system.
        Activities are performed during the pre-delivery stage as well as the post-delivery stage
        (ISO/IEC 14764:2006)
    \end{enumerate}
\end{enumerate}


Дополнительные понятия:
\begin{enumerate}
    \item Адаптивное сопровождение (adaptive maintenance)
    \begin{enumerate}
        \item Изменение (модификация) программного продукта после поставки, обеспечивающее его работоспособность в измененных или изменяющихся условиях (среде).
        (ГОСТ Р ИСО/МЭК 14764--2002)
        \item the modification of a software product, performed after delivery, to keep a software product usable in a changed or changing environment
        (ISO/IEC 14764:2006)
    \end{enumerate}

    \item Корректирующее сопровождение (corrective maintenance):
    \begin{enumerate}
        \item Реактивное изменение программного продукта, выполняемое после его поставки для корректировки обнаруженных проблем (несоответствий, ошибок).
        (ГОСТ Р ИСО/МЭК 14764--2002)
        \item the reactive modification of a software product performed after delivery to correct discovered problems
        (ISO/IEC 14764:2006)
    \end{enumerate}

    \item Полное сопровождение (perfective maintenance)
    \begin{enumerate}
        \item Модификация программного продукта после поставки для повышения его рабочих характеристик или улучшения сопровождаемости.
        (ГОСТ Р ИСО/МЭК 14764--2002)
        \item the modification of a software product after delivery to detect and correct latent faults in the software product before they are manifested as failures
        (ISO/IEC 14764:2006)
    \end{enumerate}

    \item Профилактическое сопровождение (preventive maintenance)
    \begin{enumerate}
        \item Модификация программного продукта после поставки в целях обнаружения и корректировки имеющихся в нем скрытых ошибок для предотвращения явного проявления этих ошибок при эксплуатации данного продукта.
        (ГОСТ Р ИСО/МЭК 14764--2002)
        \item the modification of a software product after delivery to detect and correct latent faults in the software product before they become operational faults
        (ISO/IEC 14764:2006)
    \end{enumerate}

    \item Отчет о проблеме (ОП) (problem report [PR])
    \begin{enumerate}
        \item Термин, используемый для определения и описания проблем, обнаруженных в программном продукте.
        (ГОСТ Р ИСО/МЭК 14764--2002)
        \item a term used to identify and describe problems detected in a software product
        (ISO/IEC 14764:2006)
    \end{enumerate}

    \item Предложение о модификации (ПР)
    \begin{enumerate}
        \item Общий термин, используемый для определения предполагаемых изменений в сопровождаемом программном продукте.
        Может быть далее классифицировано как коррекция (correction) или модернизация (enhancement)
        (ГОСТ Р ИСО/МЭК 14764--2002)
        \item a generic term used to identify proposed modifications to a software product that is being maintained
        (ISO/IEC 14764:2006)
    \end{enumerate}

    \item План сопровождаемости (maintainability plan)
    \begin{enumerate}
        \item Документ, излагающий соответствующие методы обеспечения сопровождаемости, описывающий необходимые для этого ресурсы и работы применительно к программным средствам.
        (ГОСТ Р ИСО/МЭК 14764--2002)
        \item the scope of software maintenance, the designation of who will provide maintenance, an estimate of maintenance costs.
        Guidelines for developing a Maintenance Plan
        (ISO/IEC 14764:2006)
    \end{enumerate}

    \item План сопровождения (maintenance plan)
    \begin{enumerate}
        \item Документ, излагающий соответствующие методы сопровождения, описывающий необходимые ресурсы и работы применительно к сопровождению программного продукта.
        (ГОСТ Р ИСО/МЭК 14764--2002)
        \item documented activities and tasks which should be performed and used while product maintenance
        (ISO/IEC 14764:2006)
    \end{enumerate}

    \item Программа сопровождения (maintenance program)
    \begin{enumerate}
        \item Организационная структура, обязанности, процедуры, процессы и ресурсы, используемые при выполнении плана сопровождения.
        (ГОСТ Р ИСО/МЭК 14764--2002)
        \item roles and responsibilities of the maintainer.
        (ISO/IEC 14764:2006)
    \end{enumerate}

    \item Среда программной инженерии (СПИ)
    \begin{enumerate}
        \item Набор автоматических инструментальных средств, программно-аппаратных и технических средств, необходимых для выполнения объема работ по программной инженерии.
        (ГОСТ Р ИСО/МЭК 14764--2002)
        \item the tools to initially develop and modify the software products.
        (ISO/IEC 14764:2006)
    \end{enumerate}

    \item Среда тестирования программного средства (СТПС)
    \begin{enumerate}
        \item Вспомогательное оборудование, технические и программные средства, программы, реализованные техническими средствами, процедуры и документы, необходимые для проведения квалификационных, а возможно, и других испытаний (тестирований) программного средства.
        (ГОСТ Р ИСО/МЭК 14764--2002)
        \item tools required by the methods, policies, guidelines, and standards of software maintenance activities.
        (ISO/IEC 14764:2006)
    \end{enumerate}

    \item Передача программного средства (software transition)
    \begin{enumerate}
        \item Контролируемая и координируемая последовательность действий, в процессе реализации которой разработанное программное средство передают из организации-разработчика в организацию, выполняющую его сопровождение.
        (ГОСТ Р ИСО/МЭК 14764--2002)
        \item a controlled and coordinated sequence of actions wherein software development passes from the organization performing initial software development to the organization performing software maintenance
        (ISO/IEC 14764:2006)
    \end{enumerate}
\end{enumerate}