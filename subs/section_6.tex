%! Author = vladislav.yaroshchuk
%! Date = 10/10/2021

\pagebreak


\section{Разработка регламента работы с документами.}

\subsection{Участники процесса сопровождения}

\begin{itemize}
    \item Старший сопровождающий (СТС) -- отвественное за продукт лицо
    \item Сопровождающий (С) -- представитель продукта, отвественное за продукт или часть продукта лицо
    \item Пользователь (П) -- лицо, использующее продукт
    \item Разработчик (РЗ) -- лицо, вносящее изменения в продукт
\end{itemize}

Примечание: так как за основу взят процесс сопровождения с исходным кодом, одно лицо может совмещать все четыре роли, приведенные выше.

\subsection{Регламент работы с документами}

\begin{center}
    \begin{longtable}{|p{3cm}|p{3.3cm}|p{3cm}|p{3cm}|p{3cm}|}
        \caption{Регламент работы с документами} \\
        \hline
        Документ                     & Инициирование & Создание      & Согласование \par и утверждение & Использование \\
        \hline
        План сопровождения           & СТС           & С, СТС        & СТС                             & П, РЗ, С, СТС \\
        \hline
        Отчет о проблеме             & П, РЗ, С, СТС & П, РЗ, С, СТС & С, СТС                          & П, РЗ, С, СТС \\
        \hline
        Запрос об улучшении          & П, РЗ, С, СТС & П, РЗ, С, СТС & С, СТС                          & П, РЗ, С, СТС \\
        \hline
        Описание модификации         & РЗ, С, СТС    & РЗ, С, СТС    & С, СТС                          & П, РЗ, С, СТС \\
        \hline
        Отчет об анализе модификации & С, СТС        & С, СТС        & С, СТС                          & П, РЗ, С, СТС \\
        \hline
        Отчет об изменениях в версии & С, СТС        & С, СТС        & СТС                             & П, РЗ, С, СТС \\
        \hline
    \end{longtable}
\end{center}