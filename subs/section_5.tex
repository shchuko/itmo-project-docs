%! Author = vladislav.yaroshchuk
%! Date = 10/10/2021

\pagebreak


\section{Формирование требований к содержанию и оформлению, разработка шаблонов документов и примеров их заполнения}

Требования к документам были сформированы на основе ГОСТ Р ИСО/МЭК 14764--2002\cite{gost14764}, ISO/IEC 14764:2006\cite{iso14764}, а также документации проекта QEMU\cite{qemu,qemucontribute}.

\begin{itemize}
    \item План сопровождения.
    \begin{enumerate}
        \item введение:
        \begin{enumerate}
            \item описание сопровождаемой системы;
        \end{enumerate}
        \item концепция сопровождения:
        \begin{enumerate}
            \item описание концепции;
            \item пример практического применения процесса сопровождения;
        \end{enumerate}

        \item организационные работы и работы по сопровождению:
        \begin{enumerate}
            \item обязанности сопроводителя;
            \item обязанности разработчика;
            \item обязанности пользователя;
        \end{enumerate}

        \item ресурсы:
        \begin{enumerate}
            \item программные средства разработки и сопровождения;
            \item технические средства разработки и сопровождения;
            \item шаблоны документов:
            \begin{enumerate}
                \item Отчет о проблеме
                \item Запрос об улучшении
                \item Описание модификации
                \item Отчет об анализе модификации
                \item Отчет об изменениях в версии
            \end{enumerate}
            \item дополнительные ресурсы (при необходимости):
        \end{enumerate}

        \item правила публикации изменений продукта в новых версиях;
    \end{enumerate}

    \item Отчет о проблеме -- публикуется на ресурсе отслеживания проблем~\cite{qemubugtracker}.
    Оформляется согласно шаблону, приведенному в Приложении 1
    \begin{enumerate}
        \item информация об авторе
        \item название проблемы
        \item цифровой идентификатор
        \item описание окружения, где проблема проявилась
        \item описание версии продукта
        \item конфигурация продукта
        \item описание проблемы
        \item шаги для воспроизведения проблемы
        \item дополнительная информация (если необходимо)
    \end{enumerate}

    \item Запрос об улучшении -- публикуется на ресурсе отслеживания проблем~\cite{qemubugtracker}.
    Офомляется согласно шаблону, приведенному в Приложении 2
    \begin{enumerate}
        \item информация об авторе
        \item название улучшения
        \item цифровой идентификатор
        \item цель улушения
        \item технические детали улучшения
        \item дополнительная информация (если необходимо)
    \end{enumerate}

    \item Описание модификации -- публикуется на ресурсе совместной разработки ПО~\cite{qemusrchosting,qemupatchew}.
    \begin{enumerate}
        \item название модификации
        \item информация об авторе модификации
        \item цифровой идентификатор ассоциированного отчета о проблеме (если существует)
        \item цифровой идентификатор ассоциированного запроса об улучшении (если существует)
        \item текстовое описение модификации
        \item список изменений (серия исправлений в исходном коде)
    \end{enumerate}


    \item Отчет об анализе модификации -- публикуется на ресурсе совместной разработки ПО~\cite{qemupatchew}.
    \begin{enumerate}
        \item информация об авторе;
        \item текстовое описание мнения автора;
        \item комментарии к изменениям:
        \begin{enumerate}
            \item запросы на исправления ошибок в изменениях со ссылками на места изменений
            \item запросы на улучшения в изменених со ссылками на места изменений
            \item запросы на разъяснения изменений со ссылками на места изменений
        \end{enumerate}
        \item причина отказа в принятии модификации (если существует)
    \end{enumerate}


    \item Отчет об изменениях в версии~\cite{qemu}.
    \begin{enumerate}
        \item список исправлений ошибок
        \item список улучшений
        \item список исправлений ошибок и улучшений, не вошедших в версию
        \item список неисправленных ошибок
    \end{enumerate}
\end{itemize}

Требования к оформлению Плана сопровождения и Отчета об изменениях в версии:
\begin{itemize}
    \item Шрифт: Times New Roman;
    \item Цвет текста: черный;
    \item Кегль:
    \begin{itemize}
        \item Заголовок 1: 16 пт;
        \item Заголовок 2: 16 пт;
        \item Заголовок 3: 16 пт;
        \item Заголовок 4: 16 пт;
        \item Обычный текст: 14 пт.
    \end{itemize}
    \item Междустрочный интервал: 1,5;
    \item Отступы перед и после абзаца 1,25 см:
    \item Поля:
    \begin{itemize}
        \item Сверху: 20 мм;
        \item Справа: 20 мм;
        \item Снизу: 20 мм.
        \item Слева: 30 мм;
    \end{itemize}
    \item Выравнивание по ширине;
    \item Для выделения в тексте использовать полужирный шрифт;
    \item Для терминов использование курсивный шрифт;
\end{itemize}

Остальные документы (Отчет о проблеме, Запрос об улучшении, Описание модификации) форматируются согласно приведенным Markdown-шаблонам в Приложениях.
